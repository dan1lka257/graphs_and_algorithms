\documentclass[a4paper, 12pt]{article}
\usepackage[english, russian]{babel}
\usepackage{svg}
\usepackage[T2A]{fontenc}
\usepackage[utf8]{inputenc}
\usepackage{ stmaryrd }
\usepackage{ dsfont, amsmath, amsfonts, amssymb, amsthm, mathtools }
\theoremstyle{plain}
\usepackage{listings}
\usepackage{xcolor}
\newtheorem{theorem}{Теорема}
\newtheorem{lemma}{Лемма}
\newtheorem{definition}{Определение}
\usepackage[english, russian]{babel}
\usepackage[T2A]{fontenc}
\usepackage[utf8]{inputenc}
\usepackage{amssymb}
\usepackage{amsmath}
\usepackage{amsfonts}
\usepackage{geometry}
\geometry{top=15mm}
\geometry{bottom=15mm}
\geometry{left=15mm}
\geometry{right=15mm}
\linespread{1.3}
\usepackage{graphicx}
\usepackage{graphicx}
\usepackage{setspace}
\usepackage{amsmath}
\usepackage{tikz}
\usetikzlibrary{graphs}
\usepackage{amscd}
\usepackage[all]{xy}
\definecolor{codegreen}{rgb}{0,0.6,0}
\definecolor{codegray}{rgb}{0.5,0.5,0.5}
\definecolor{codepurple}{rgb}{0.58,0,0.82}
\definecolor{backcolour}{rgb}{0.95,0.95,0.92}
\lstdefinestyle{mystyle}{
	backgroundcolor=\color{backcolour},   
	commentstyle=\color{codegreen},
	keywordstyle=\color{magenta},
	numberstyle=\tiny\color{codegray},
	stringstyle=\color{codepurple},
	basicstyle=\ttfamily\footnotesize,
	breakatwhitespace=false,         
	breaklines=true,                 
	captionpos=b,                    
	keepspaces=true,                 
	numbers=left,                    
	numbersep=5pt,                  
	showspaces=false,                
	showstringspaces=false,
	showtabs=false,                  
	tabsize=2
}

\lstset{style=mystyle}

\begin{document}
	\textsf{Hello everyone!}
	\section*{Вступление}
	\subsection*{Тема}
	Генерация 3D-изображений дискретных динамических систем, заданных на поверхностях, из трехцветных графов.
	\subsection*{План работы}
	В ходе работы предполагается создать программу, которая генерирует различные трехцветные графы, проверяет их допустимость как инварианта для диффеоморфизма, заданного на поверхности, и рисует 3D-изображение с каскадом, соответствующим сгенерированному графу.
	\section*{Трёхцветный граф}
	В этой главе будет представлено построение трёхцветного графа по градиентно-подобному каскаду на поверхности. Стоит отметить, что на языке трёхцветных графов получена полная топологическая классификация градиентно-подобных каскадов на поверхностях.
	\subsection*{Трёхцветный граф как полный топологический инвариант диффеоморфизма на поверхности}
	\begin{definition}
		Диффеоморфизм $f: M^n \shortrightarrow M^n$, заданный на гладком замкнутом n-многообразии, называется диффеоморфизмом Морса-Смейла, если:$\\$1) неблуждающее множество $\Omega_f$ гиперболично и конечно (т.е. состоит из конечного чила периодических точек, для которых модули собственных значений матрицы Якоби не равны единице);$\\$2) для любых периодических точек p, q устойчивое многообразие $W^s_p$ и неустойчивое многообразие $W^u_q$ либо не пересекаются, либо трансверсальны в каждой точке пересечения. $\\$
	\end{definition}
	Пусть $f: M^n \shortrightarrow M^n$ - диффеоморфизм Морса-Смейла, тогда периодические точки называются источниками, если неустойчивое многообразие $W^u_q$ имеет размерность $n$, стоками, если $0$, и сёдлами при остальных. $\\$
	Далее скажем, что для любой периодической точки $p$ диффеоморфизма $f$ компоненты связности $W^s_p \textbackslash p~(W^u_p \textbackslash p)$ называются её устойчивыми (неустойчивыми) сепаратрисами. $\\$
	Рассмотрим класс диффеоморфизмов на поверхности $M^2$, тогда диффеоморфизм Морса-Смейла называется градиентно-подобным, если $W^s_p \cap W^u_p = \o$ для любых различных седловых точек $p,q$. $\\$
	Удалим из поверхности $M^2$ замыкание объединения устойчивых и неустойчивых многообразий седловых точек $f$ и получим множество $M' = M^2 \textbackslash (W^{u}_{\Omega^0_f} \cup W^{u}_{\Omega^1_f} \cup W^{s}_{\Omega^1_f} \cup W^{s}_{\Omega^2_f})$. $\\$
	$M'$ является объединением ячеек, гомеоморфных открытому двумерному диску, граница которых имеет один из 3-х следующих видов: (картинка с ячейками) $\\$
	Пусть $A$ - ячейка из $M'$
	\begin{definition}
		Конечным графом называется упорядоченная пара (B,E), для которой выполнены следующие условия: $\\$
		1) B - непустое конечное множество вершин $\\$
		2) E - множество пар вершин, называемых рёбрами $\\$
	\end{definition}
	\begin{definition}
		Если граф содержит ребро e = (a,b), то каждую из вершин a, b называют инцидентной ребру e и говорят, что вершины a и b соединены ребром e.$\\$
	\end{definition}
	\begin{definition}
		Путём в графе называют конечную последовательность его вершин и рёбер вида: $b_0, (b_0, b_1), b_1, \dots, b_{i-1}, (b_{i-1}, b_{i}), b_{i}, \dots, b_{k-1}, (b_{k-1}, b_{k}), b_{k}, k >= 1$. Число k называется длиной пути, оно совпадает с числом входящих в него рёбер.$\\$
	\end{definition}
	\begin{definition}
		Граф называют связным, если любые две его вершины можно соединить путём.$\\$
	\end{definition}
	\begin{definition}
		Циклом длины $k \in \mathds{N}$ в графе называют конечное подмножество его вершин и рёбер вида \{$b_0, (b_0, b_1), b_1, \dots, b_{i-1}, (b_{i-1}, b_{i}), b_{i}, \dots, b_{k-1}, (b_{k-1}, b_{0})$\}. Простым циклом называют цикл, у которого все вершины и рёбра попарно различны.$\\$
	\end{definition}
	\begin{definition}
		Граф T называется трёхцветным графом, если: $\\$
		1) множество рёбер графа T является объединением трёх подмножеств, каждое из которых состоит из трёх рёбер одного и того же определенного цвета (цвета рёбер из разных подмножеств не совпадают, будем обозначать эти цвета буквами s, t, u, а рёбра для краткости будем называть s-, t-, u-рёбрами); $\\$
		2) каждая вершина графа T инцидентна в точности трём рёбрам различных цветов;$\\$
		3) граф не содержит циклов длины 1.$\\$
	\end{definition}
	\begin{definition}
		Простой цикл трёхцветного графа T назовём двухцветным циклом типа su, tu или st, если он содержит рёбра в точности двух цветов s и u, t и u, s и t соответственно. Непосредственно из определения трёхцветного графа следует, что длина любого двухцветного цикла является чётным числом (так как цвета рёбер строго чередуются), а отношение на множестве вершин, состоящее в принадлежности двёхцветному циклу определённого типа, является отношением эквивалентности. $\\$
	\end{definition}
	\begin{definition}
		Построим трёхцветный граф $T_f$, соответствующий диффеоморфизму $f \in G$, следующим образом:$\\$
		1) вершины графа $T_f$ взаимно однозначно соответствуют треугольным областям множества $\Delta$; $\\$
		2) две вершины графа инцидентны ребру цвета s, t, u, если соответствующие этим вершинам треугольные области имеют общую s-, t- или u-кривую  $\\$
		$\\$
		Граф $T_f$ полностью удовлетворяет определению трёхцветного графа $\\$
	\end{definition}
	\begin{theorem}
		Теорема 1. Для того чтобы диффеоморфизмы f, f' из класса G были топологически сопряжены, необходимо и достаточно, чтобы их графы ($T_f, P_f$) и ($T_{f'}, P_{f'}$) были изоморфны $\\$
	\end{theorem}
	\begin{definition}
		Определение 2. Трёхцветный граф (T,P) назовём допустимым, если он обладает следующими свойствами: $\\$
		1) граф T связен;$\\$
		2) длина любого su-цикла графа T равна 4; $\\$
		3) автоморфизм P является периодическим. $\\$
	\end{definition}
	\begin{lemma}
		Пусть $f \in G$. Тогда трёхцветный граф ($T_f, P_f$) является допустимым. $\\$
	\end{lemma}
	\begin{theorem}
		Пусть (T,P) - допустимый трёхцветный граф. Тогда существует диффеоморфизм $f:M^2 \shortrightarrow M^2$ из класса G, граф ($T_f, P_f$) которого изоморфен графу (T,P). При этом: $\\$
		1) эйлерова характеристика поверхности $M^2$ вычисляется по формуле $X(M^2) = v_0 - v_1 + v_2$, где $v_0, v_1, v_2$ - число всех tu-, su-, st-циклов графа T соответственно; $\\$
		2) поверхность $M^2$ ориентируема тогда и только тогда, когда все циклы графа T имеют чётную длину
	\end{theorem}
	\subsection*{title2}
	\subsection*{title3}
	\section{title4}
	
	Here is a general recipe for a polynomial whose level set is an $n$-torus in $\mathbb R^3$.
	
	First, take the polynomial $$\begin{align}f(x) &= \prod_{i=1}^n (x-(i-1))(x-i) \\ &= x(x-1)^2(x-2)^2\cdots(x-(n-1))^2(x-n)\end{align}$$ which is positive as $x\to\pm\infty$, crosses zero at $x=0$ and $x=n$, and touches zero from below at $i = 1, 2, \ldots, n-1$. Examples: $n=1$, $n=2$, $n=5$.
	
	Then let $$g(x,y) = f(x) + y^2,$$ so that the set of points $g(x,y)=0$ forms $n$ connected loops ($n=1$, $n=2$, $n=5$). Finally, define $$h(x,y,z) = g(x,y)^2 + z^2 - r^2,$$ which "inflates" the loops in three dimensions. For small enough $r$, the level set $h(x,y,z) = 0$ is an $n$-torus. For example, here's $n=2$ and $r=0.1$, for which the zero level set of $h(x,y,z) = \left(x(x-1)^2(x-2)+y^2\right)^2+z^2 - 0.01$ is plotted:
	
	Here's another way to obtain a "double torus": you can start from the implicit equation of a lemniscate, which is a curve shaped like a figure-eight. One could, for instance, choose to use the lemniscate of Gerono:
	
	$$x^4-a^2(x^2-y^2)=0$$
	
	or the hyperbolic lemniscate, which is the inverse curve of the hyperbola:
	
	$$(x^2+y^2)^2-a^2x^2+b^2y^2=0$$
	
	(the famous lemniscate of Bernoulli is a special case of this, corresponding to the inversion of an equilateral hyperbola).
	
	Now, to generate a double torus from these lemniscates, if you have the implicit Cartesian equation in the form $F(x,y)=0$, you can perform the "inflation" step of Rahul's approach; that is, form the equation
	
	$$F(x,y)^2+z^2=\varepsilon$$
	
	where $\varepsilon$ is a tiny number.
	
	For instance, here's a double torus formed from the lemniscate of Bernoulli: $$((x^2+y^2)^2-x^2+y^2)^2+z^2=\frac1{100}$$
	
\end{document}